
\chapter {Introduction}

The \egraph data structure has been the focus of programming language and verification research for a long time. 
First invented in the 1980s, it is originally designed for automated theorem proving (ATP) and is extensively used in modern theorem provers \citep{eqsatthesis, efficient-ematching, proof-producing,simplify,Z3,CVC4}. 
During the last decade, \egraphs are also re-purposed for program optimization, known as equality saturation \citep{eqsat, denaili, spores, semsearch,herbie, egg}. 
At its core, \egraphs generalize the union-find data structure \citep{tarjan}, which efficiently represents equivalence relations, to congruence relations. 
A congruence relation $\cong$  is an equivalence relation where two terms are also congruent  (i.e., $f(t_1,\ldots,t_k)\cong f(t'_1,\ldots,t'_k)$) if all children of the two terms are pairwise congruent (i.e., $t_i\cong t'_i$). 
\Egraphs maintains the congruence invariant by maintaining a set of equivalent classes (known as \textit{e-classes}) consisting of a set of nodes (known as \textit{e-nodes}), while every \enode consists of a head function symbol and a list of children e-classes.

\Ematching is a fundamental operation on \egraphs \citep{efficient-ematching, simplify}. 
It finds the set of terms in an \egraph matching a given pattern. 
For example, for pattern $f(\alpha, g(\alpha))$\footnote{
 $\alpha$ is a variable in this pattern. We use $a,b,c$ for constants and $\alpha, \beta, \gamma$ for variables as a convention.
}, \ematching finds every term whose head symbol is $f$ and whose second argument has a head symbol $g$ and whose first argument to $f$ and first argument to $g$ are the same.
\Ematching is a central procedure in state-of-the-art SMT solvers, including \textsc{Z3} \citep{Z3} and \textsc{CVC4} \citep{CVC4}. SMT solvers use \ematching to instantiate quantified formulas over ground terms. Moreover, \ematching is also the bottleneck of many equality saturation--based program optimizers \citep{spores,herbie,egg}. In equality saturation, the optimizer uses \ematching to match the left-hand side of the rewrite rules and fires new terms based on the matches, and \ematching usually takes 60--90\% of the overall run time.
Therefore, in both applications, the performance of e-matching is critical.

Several algorithms are proposed for e-matching \citep{efficient-ematching,simplify}.
However, to our knowledge, existing e-matching algorithms are based on na\"ive backtracking, which is suboptimal in many cases. 
In particular, their query planning does not exploit the constraints implied by multi-occurrences of the same variable, which is nonetheless a common pattern in practice.
To see this, consider the pattern $f(\alpha,g(\alpha))$ again. 
This pattern generates three constraints for a potential matching \enode $n$: 
\begin{align}
    n.\sym&=f \label{eqn:constraint1}\\
    n.\child_2.\sym&=g \label{eqn:constraint2}\\
    n.\child_1&=n.\child_2.\child_1 \label{eqn:constraint3}
\end{align}
There are two kinds of constraints here. The first kind of constraint is derived from the structure of the pattern. In this example, the structure of $f(\alpha, g(\alpha))$ constrains the head symbol of all (sub)terms to be $f$ and $g$ respectively (i.e., constraint \ref{eqn:constraint1} and \ref{eqn:constraint2}), which we call {\it structural constraints}. The second kind of constraints is implied by the multi-occurrence of the same variable. Here, the occurrence of $\alpha$ implies that the terms at these positions should be equivalent with each other for all matches (i.e., constraint \ref{eqn:constraint1}), which we call {\it equality constraints}. 

To our knowledge, existing backtracking-based algorithms only exploit the structural constraint during query planning and fail to consider equality constraints in general. Therefore, running backtracking-based algorithms on the above pattern, terms violating constraint \ref{eqn:constraint3} are only pruned away \textit{a posteriori} when the partial match containing violating variables are instantiated (See Chapter~\ref{sec:background} for more details). Therefore, excessive terms like $f(a, g(b))$ for $a\not\cong b$ will be enumerated during search and only pruned away before yielding. 

We solve this issue by taking a relational view of \egraphs and \ematching, which we call \textit{relational e-matching}. Under this view, \egraphs correspond to a relational database and \ematching patterns correspond to a restricted class of relational queries, known as {\it conjunctive queries}. One key advantage of this relational view is that the conjunctive query representation provides a unified way to explicitly encode \textit{both} structural constraints and equality constraints, of which the conjunctive query solver could take advantage during query planning. 
% In particular, the mapped CQ can be classified into three categories: linear acyclic queries, non-linear acyclic queries, and cyclic queries.

One particular kind of conjunctive queries practical \ematching patterns may correspond to is cyclic queries, where query plans based on traditional two-way joins, including hash joins and merge-sort joins, are known to be suboptimal \citep{agm}.
On cyclic queries, two-way joins cannot leverage all the constraints to prune the search space at once, so candidates that does not satisfy all the constraints will be enumerated.
Therefore, we take inspirations from the theoretical advances in database theory community and utilize generic join algorithm as the conjunctive query solver. 
Generic join algorithm is a worst-case optimal algorithm on conjunctive queries, and it avoids to enumerate candidates that do not satisfy all the constraints.
Moreover, it is particularly efficient with complex queries (e.g., cyclic queries). 
Using generic joins, our \ematching algorithm preserves the guarantees of worst-case optimality. 
Finally, we instantiate our algorithm in \egg \citep{egg}, a program optimization framework based on equality saturation. We evaluate our implementation using some microbenchmarks, which shows asymptotic speedup and order of magnitude performance gain.

In summary, we made the following contributions:
\begin{itemize}
    \item We propose relational e-matching, a worst-case optimal and efficient approach to \ematching by running generic joins on \ematching-translated conjunctive queries over the relational representations of \egraphs.
    \item We instantiate relational e-matching in an efficient implementation integrates to \textsc{egg} and evaluate it on several benchmarks.
\end{itemize}

% pp-pair is somehow a way to work with equality constraint under incremental setting.

